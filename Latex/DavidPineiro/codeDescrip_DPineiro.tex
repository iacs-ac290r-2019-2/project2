We used various methods in order to simulate the drug being dissipated through the artery stenosis. 

We first created the shape of the artery using a Python package contained in the file ShapePainter.py which allows us to specify the shape of the vein. This file creates a skeletal mapping of the lattice which is later interpreted by MUPHY another package performing the simulation. ShapePainter.py a .hdr file, which contains the mesh header, a .dat file which details information regarding the mesh nodes, and a .ios file which corresponds to the inlet and outlet boundary conditions which are specified in this file. 

These output files are then interpreted by another package called MUPHY. This package works well for complex and deformed physiological geometries such as veins and arteries, and can cover various
space/time scales with multi-resolution, often useful for physical systems with various scopes. MUPHY is used via MagicUniverse which instantiates a MagicBegins() object that creates an environment for where the LBM computation will take place. There, various fluid objects can be created with specific properties regarding their viscosity among others. 